% Compile with xelatex specimen.tex
\documentclass{article}
\usepackage[silent]{fontspec}
\setmainfont[Scale=1,Ligatures=Required]{Chaucer.otf}
\usepackage{kantlipsum}
\title{Chaucer font}
\author{epilys}
\begin{document}
\maketitle
This font is based on William Morris' Chaucer type used in the Kelmscott Chaucer edition.

\section{Sample text}

\subsection{\textsection{}\thinspace{}THE KNIGHT'S TALE}
\begin{verse}
WHILOM, as olde stories tellen us,\\
There was a duke that highte Theseus.\\
Of Athens he was lord and governor,\\
And in his time such a conqueror\\
That greater was there none under the sun.\\
Full many a riche country had he won.\\
What with his wisdom and his chivalry,\\
He conquer'd all the regne of Feminie,\\
That whilom was y-cleped Scythia;\\
And weddede the Queen Hippolyta\\
And brought her home with him to his country\\
With muchel glory and great solemnity,\\
And eke her younge sister Emily,\\
And thus with vict'ry and with melody\\
Let I this worthy Duke to Athens ride,\\
And all his host, in armes him beside.
\end{verse}

\begin{verse}
\textpilcrow{}\thinspace{}And certes, if it n'ere too long to hear,\\
I would have told you fully the mannere,\\
How wonnen was the regne of Feminie,\\
By Theseus, and by his chivalry;\\
And of the greate battle for the nonce\\
Betwixt Athenes and the Amazons;\\
And how assieged was Hippolyta,\\
The faire hardy queen of Scythia;\\
And of the feast that was at her wedding\\
And of the tempest at her homecoming.\\
But all these things I must as now forbear.\\
I have, God wot, a large field to ear\\
And weake be the oxen in my plough;\\
The remnant of my tale is long enow.\\
I will not letten eke none of this rout.\\
Let every fellow tell his tale about,\\
And let see now who shall the supper win.\\
There as I left, I will again begin.
\end{verse}

\begin{verse}
\textpilcrow{}\thinspace{}This Duke, of whom I make mentioun,\\
When he was come almost unto the town,\\
In all his weal, and in his moste pride,\\
He was ware, as he cast his eye aside,\\
Where that there kneeled in the highe way\\
A company of ladies, tway and tway,\\
Each after other, clad in clothes black:\\
But such a cry and such a woe they make,\\
That in this world n'is creature living,\\
That hearde such another waimenting\\
And of this crying would they never stenten,\\
Till they the reines of his bridle henten.\\
"What folk be ye that at mine homecoming\\
Perturben so my feaste with crying?"\\
Quoth Theseus; "Have ye so great envy\\
Of mine honour, that thus complain and cry?\\
Or who hath you misboden, or offended?\\
Do telle me, if it may be amended;\\
And why that ye be clad thus all in black?"
\end{verse}
\end{document}
