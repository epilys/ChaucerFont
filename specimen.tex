% Compile with xelatex specimen.tex
\documentclass{article}
\usepackage[landscape,a4paper]{geometry}%
\usepackage[silent]{fontspec}
\setmainfont[Scale=1,Ligatures=Required]{Chaucer.otf}
\usepackage{multicol}
\usepackage{xcolor}
\definecolor{sidered}{RGB}{185, 59, 21}
\title{Chaucer typeface}
\author{epilys}
\begin{document}
\maketitle
This font is based on William Morris' Chaucer type used in the Kelmscott Chaucer edition.
\section*{Sample text}%
\reversemarginpar{}%
\marginpar{\textcolor{sidered}{\\ \\ The\\ Knyghtes\\ Tale}}%
\begin{multicols}{2}%
\noindent{\LARGE{}W}HILOM, as olde stories tellen us,\\
Ther was a duc that highte Theseus;\\
Of Atthenes he was lord and governour,\\
And in his time such a conqueror\\
That greater was there none under the sun.\\
Full many a riche country had he won.\\
What with his wisdom and his chivalry,\\
He conquer'd all the regne of Feminie,\\
That whilom was ycleped Scythia;\\
And weddede the queene Ypolita,\\
And broghte hire hoom with hym in his contree,\\
With muchel glorie and greet solemnytee,\\
And eek her younge sister Emelye.\\
And thus with victorie and with melodye\\
Let I this worthy Duc to Atthenes ryde,\\
And al his hoost, in armes hym bisyde.
\columnbreak%

\noindent\textpilcrow{}\thinspace{}And certes, if it nere to long to heere,\\
I wolde have toold yow fully the manere,\\
How wonnen was the regne of Feminye,\\
By Theseus, and by his chivalrye;\\
And of the grete bataille for the nonce\\
Betwixen Atthenes and Amazones;\\
And how asseged was Ypolita,\\
The faire hardy queene of Scithia;\\
And of the feast that was at hir weddynge,\\
And of the tempest at hir hoom comynge;\\
But al that thyng I moot as now forbere.\\
I have, God woot, a large feeld to ere,\\
And wayke been the oxen in my plough.\\
The remenant of the tale is long ynough.\\
I wol nat letten eek noon of this route;\\
Lat every felawe telle his tale aboute,\\
And lat se now who shal the soper wynne,\\
And ther I lefte, I wol ageyne bigynne.
\end{multicols}%
\end{document}
